% TEMPLATE for Usenix papers, specifically to meet requirements of
%  USENIX '05
% originally a template for producing IEEE-format articles using LaTeX.
%   written by Matthew Ward, CS Department, Worcester Polytechnic Institute.
% adapted by David Beazley for his excellent SWIG paper in Proceedings,
%   Tcl 96
% turned into a smartass generic template by De Clarke, with thanks to
%   both the above pioneers
% use at your own risk.  Complaints to /dev/null.
% make it two column with no page numbering, default is 10 point

% Munged by Fred Douglis <douglis@research.att.com> 10/97 to separate
% the .sty file from the LaTeX source template, so that people can
% more easily include the .sty file into an existing document.  Also
% changed to more closely follow the style guidelines as represented
% by the Word sample file. 

% Note that since 2010, USENIX does not require endnotes. If you want
% foot of page notes, don't include the endnotes package in the 
% usepackage command, below.

\documentclass[letterpaper,twocolumn,10pt]{article}
\usepackage{custom}
\usepackage{usenix,epsfig,endnotes,hyperref}
\usepackage{color, enumitem}
\begin{document}

%don't want date printed
\date{}

%make title bold and 14 pt font (Latex default is non-bold, 16 pt)
\title{\Large \bf Detection and Analysis of Disaster-Related Tweets}

\author{
{\rm Daniel Solomon}\\
\texttt{\red{EMAIL@mail.tau.ac.il}}
\and
{\rm Gal Ron}\\
\texttt{galr1@mail.tau.ac.il‬}
\and
{\rm Omri Ben-Horin}\\
\texttt{\red{EMAIL@mail.tau.ac.il}}
}

\maketitle

% Use the following at camera-ready time to suppress page numbers.
% Comment it out when you first submit the paper for review.
\thispagestyle{empty}


\abstract{}
TODO

%%%%%%%%%%%%%%%%%%%%%%%%%%%%%%%%%%%%%%%%%%
\section{Introduction}
The popular microblogging service Twitter is a fruitful source of user-created content. With hundreds of millions of new tweets every day, Twitter has become a probe to human behavior and opinions from around the globe. The Twitter 'corpus' reflects political and social trends, popular culture, global and local happenings, and much more. In addition, tweets are easy to access and aggregate in real-time. Therefore, we experience an increased interest in natural language processing research of Twitter data.

As one of the world's most widely used social networks, Twitter is an effective channel of communication and plays an important role during a crisis or emergency. The live stream of tweets can be used to identify reports and calls for help in emergency situations, such as accidents, violent crimes, natural disasters and terror attacks (which we all refer to as 'disasters' in this paper).

In this work we present a model trained to identify disaster-related tweets from other messages, using a natural language processing pipeline adjusted to the special features of Twitter tweets. In addition, we present two experiments conducted on disaster-related tweets. First, we separate \textit{subjective} tweets (example: \red{\tweet{TWEET}}) from \textit{objective} reports on disasters (example: \red{\tweet{TWEET}}). We also recognize named-entities in disaster-related tweets to enrich our knowledge on the disaster (mostly location).


\subsection{Tweets vs. Traditional Corpora}
Tweets have some unique features that differ from traditional corpora (such as WSJ corpus). These features should be taken into consideration when implementing natural language processing techniques.

Here's a tweet:

\tweet{RT This is an \#awsome tweet lmao :O}

\red{TODO}
	

%%%%%%%%%%%%%%%%%%%%%%%%%%%%%%%%%%%%%%%%%%
\section{Analysis Wokrflow}

\paragraph{keywords} TODO

\begin{itemize}[noitemsep,nolistsep]
	\item A
	\item B
	\item C
\end{itemize}

%%%%%%%%%%%%%%%%%%%%%%%%%%%%%%%%%%%%%%%%%%
\section{Tweet Classification}

\paragraph{keywords} TODO

%%%%%%%%%%%%%%%%%%%%%%%%%%%%%%%%%%%%%%%%%%
\section{Named-Entity Recognition in Tweets}

\paragraph{keywords} TODO

%%%%%%%%%%%%%%%%%%%%%%%%%%%%%%%%%%%%%%%%%%
\section{Experimenting with Recent Tweets}

\paragraph{keywords} Twitter's Search API

%%%%%%%%%%%%%%%%%%%%%%%%%%%%%%%%%%%%%%%%%%
\section{Conclusions}

\paragraph{Future work} TODO

%%%%%%%%%%%%%%%%%%%%%%%%%%%%%%%%%%%%%%%%%%

{\footnotesize \bibliographystyle{acm}
\bibliography{references}}

\end{document}







